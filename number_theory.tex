\documentclass[11pt]{article}
\renewcommand{\baselinestretch}{1.05}
\usepackage{amsmath,amsthm,verbatim,amssymb,amsfonts,amscd,graphicx,mathtools}
\usepackage{graphics}
\usepackage{graphicx}
\usepackage{color}
\usepackage{pgfplots}
% \usepackage{csquotes}
\usepackage{fancyvrb}
% \usepackage{wrapfig}
\usepackage{centernot}
\usepackage{pifont}
\usepackage{hyperref}
\topmargin0.0cm \headheight0.0cm
\headsep0.0cm
\oddsidemargin0.0cm
\textheight23.0cm
\textwidth16.5cm
\footskip1.0cm
\theoremstyle{plain}
\newtheorem{theorem}{Theorem}
\newtheorem{corollary}{Corollary}
\newtheorem{lemma}{Lemma}
\newtheorem{proposition}{Proposition}
\newtheorem*{surfacecor}{Corollary 1}
\newtheorem{conjecture}{Conjecture}
\newtheorem{question}{Question}
\theoremstyle{definition}
\newtheorem{definition}{Definition}

\newenvironment{packed_enum}{
\begin{enumerate}
  \setlength{\itemsep}{1pt}
  \setlength{\parskip}{0pt}
  \setlength{\parsep}{0pt}
}{\end{enumerate}}

\newenvironment{packed_items}{
\begin{itemize}
  \setlength{\itemsep}{1pt}
  \setlength{\parskip}{0pt}
  \setlength{\parsep}{0pt}
}{\end{itemize}}

\begin{document}

\section*{2. Number Theory}

\subsection*{Divisibility}

An integer $a$ is \begin{em}divisible\end{em} by the integer $b$, written 
$$b \mid a$$ 
if there exists some integer $c$, such that $ a = bc. $ Other ways to say this is that $b$ \begin{em}divides\end{em} $a$, or that $b$ is a factor, or \begin{em}divisor\end{em} of $a$. Divisibility is a binary relation (see Chapter 4) defined over $\mathbb{Z}$.

\subsubsection*{Basic properties}

%The divisibility relation satisfies the reflexive and transitive properites of binary relations, described in Chapter 4. 

Every number divides itself, since $a = a \cdot 1$ for all $a$. We just let $c = 1.$ This also means that $1$ divides every integer, in which case we set $c = a$. If $a$ divides $b,$ and $b$ divides $c,$ then there are integers $d$ and $e$, such that $b = ad$ and $c = be$. We then have $c = be = (ad)e = a(de),$ which means that $a$ divides $c.$ So if $a \mid b \text{ and } b \mid c,$ then $a \mid c.$ This property of a relation, known as \begin{em}transitivity\end{em}, is discussed more in Chapter 4. Every number is also a divisor of zero, since $0 = a \cdot 0$. 

% If $a$ divides $b,$ meaning that $b = ac$ for some $c,$ and $b$ divides $a,$ so that $a = bd$, for some $d,$ then $a = acd,$ which means that $cd = 1$ and therefore $c = d = \pm1.$
% 
% $a = b$ (antisymmetry).  




% \subsubsection*{Transitivity of divisibility relation}


\subsection*{Common Divisors and the GCD function}

If a number $a$ divides both $b$ and $c$, then $a$ is a \begin{em}common divisor\end{em} of $b$ and $c$. A number $d$ is the \begin{em}greatest common divisor\end{em}, or $\gcd,$ of $b$ and $c$, when
  
\begin{packed_items}
  \item $d$ is a common divisor of $b$ and $c$, and
  \item if $e$ is a common divisor of $b$ and $c$, then $e$ divides $d.$
\end{packed_items}

That is, the $\gcd$ is the unique common divisor of two numbers which is divisible by any common divisor of those two numbers. For example, the divisors of $42$ are $\{ 1,2,3,6,7,14,21,42 \}$ and the divisors of $24$ are $\{ 1,2,3,4,6,8,12,24 \}$. The common divisors of $42$ and $24$ are the elements in the intersection of these two sets, viz., $\{ 1,2,3,6 \}$. Every element of this set is a divisor of $6$. So the $\gcd$ of $42$ and $24$ is $6.$

\subsubsection*{Prime numbers and coprimality}

A number $p \ge 2$ whose only divisors are $1$ and $p$ itself is called a \begin{em}prime\end{em} number.

Two integers whose $\gcd$ is equal to $1$ are called \begin{em}coprime\end{em}, or relatively prime.

For any given sequence of numbers $\{a_1, a_2, \dots, a_n\}$ we can find a number $q$ that is relatively prime to every number $a_i$ in the sequence.


First note that if $a \mid b$ and $a \mid c$, then $a \mid (b - c)$.

then b = ax and c = ay, so $b - c = ax - ay = a(x - y)$ 

\subsection*{Common Multiples and the LCM function}

\subsection*{Euclidean Division}

\subsection*{Basis Representation Theorem}

\subsection*{Congruence and Modular Arithmetic}

\subsection*{Euler's Totient Function}

Euler's totient function, usually written as $\varphi(n)$, counts the number of positive integers up to $n$ that are coprime (relatively prime) to $n$.

$$ \varphi(n) = |\{k \in \mathbb{N} : \gcd(k, n) = 1\}| $$

\begin{center}
  \begin{tabular}{l l l}
    $n$ & $\{k : \gcd(k, n1) = 1\}$ & $\varphi(n)$ \\
    \hline
    1  & \{1\}                    & 1 \\
    2  & \{1\}                    & 1 \\
    3  & \{1,2\}                  & 2 \\
    4  & \{1,3\}                  & 2 \\
    5  & \{1,2,3,4\}              & 4 \\
    6  & \{1,5\}                  & 2 \\
    7  & \{1,2,3,4,5,6\}          & 6 \\
    8  & \{1,3,5,7\}              & 4 \\
    9  & \{1,2,4,5,7,8\}          & 6 \\
    10 & \{1,3,7,9\}              & 4 \\
    11 & \{1,2,3,4,5,6,7,8,9,10\} & 10 \\
    12 & \{1,5,7,11\}             & 4 \\
  \end{tabular}
\end{center}

\subsection*{Fermat's Little Theorem}

$$
a^p \equiv a (\mod p)
$$

for every prime number $p$ and integer $a$.

\subsection*{Chinese Remainder Theorem}

\end{document}

