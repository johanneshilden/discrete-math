\documentclass[11pt]{article}
\renewcommand{\baselinestretch}{1.05}
\usepackage{amsmath,amsthm,verbatim,amssymb,amsfonts,amscd,graphicx,mathtools}
\usepackage{graphics}
\usepackage{graphicx}
\usepackage{color}
\usepackage{pgfplots}
% \usepackage{csquotes}
\usepackage{fancyvrb}
% \usepackage{wrapfig}
\usepackage{centernot}
\usepackage{pifont}
\usepackage{hyperref}
\topmargin0.0cm \headheight0.0cm
\headsep0.0cm
\oddsidemargin0.0cm
\textheight23.0cm
\textwidth16.5cm
\footskip1.0cm
\theoremstyle{plain}
\newtheorem{theorem}{Theorem}
\newtheorem{corollary}{Corollary}
\newtheorem{lemma}{Lemma}
\newtheorem{proposition}{Proposition}
\newtheorem*{surfacecor}{Corollary 1}
\newtheorem{conjecture}{Conjecture}
\newtheorem{question}{Question}
\theoremstyle{definition}
\newtheorem{definition}{Definition}

\newenvironment{packed_enum}{
\begin{enumerate}
  \setlength{\itemsep}{1pt}
  \setlength{\parskip}{0pt}
  \setlength{\parsep}{0pt}
}{\end{enumerate}}

\newenvironment{packed_items}{
\begin{itemize}
  \setlength{\itemsep}{1pt}
  \setlength{\parskip}{0pt}
  \setlength{\parsep}{0pt}
}{\end{itemize}}

\begin{document}

\section*{2. Number Theory}

\subsection*{Divisibility}

An integer $a$ is \begin{em}divisible\end{em} by the integer $b$, written $b \mid a$, if there exists an integer $c$, such that $ a = bc. $ Divisibility is a binary relation (See Chapter 4).

\subsubsection*{Basic properties}

\subsubsection*{Transitivity of divisibility relation}

If $a$ divides $b$, and $b$ divides $c$ then there are integers $d, e$ such that $b = ad$ and $c = be$. Then, $ c = be = (ad)e = a(de) $ which means that $a$ divides $c$. In other words,
$$ a \mid b \text{ and } b \mid c \implies a \mid c.  $$

This is the transitivity property of binary relations, described in Chapter 4.

\subsection*{Euclidean Division}

\subsection*{Common Divisors and the GCD function}

\subsection*{Basis Representation Theorem}

\subsection*{Congruence and Modular Arithmetic}

\subsection*{Euler's Totient Function}

Euler's totient function, usually written as $\varphi(n)$, counts the number of positive integers up to $n$ that are coprime (relatively prime) to $n$.

$$ \varphi(n) = |\{k \in \mathbb{N} : \gcd(k, n) = 1\}| $$

\begin{center}
  \begin{tabular}{l l l}
    $n$ & $\{k : \gcd(k, n1) = 1\}$ & $\varphi(n)$ \\
    \hline
    1  & \{1\}                    & 1 \\
    2  & \{1\}                    & 1 \\
    3  & \{1,2\}                  & 2 \\
    4  & \{1,3\}                  & 2 \\
    5  & \{1,2,3,4\}              & 4 \\
    6  & \{1,5\}                  & 2 \\
    7  & \{1,2,3,4,5,6\}          & 6 \\
    8  & \{1,3,5,7\}              & 4 \\
    9  & \{1,2,4,5,7,8\}          & 6 \\
    10 & \{1,3,7,9\}              & 4 \\
    11 & \{1,2,3,4,5,6,7,8,9,10\} & 10 \\
    12 & \{1,5,7,11\}             & 4 \\
  \end{tabular}
\end{center}

\subsection*{Fermat's Little Theorem}

$$
a^p \equiv a (\mod p)
$$

for every prime number $p$, and integer $a$.

\subsection*{Chinese Remainder Theorem}

\end{document}

